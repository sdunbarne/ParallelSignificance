%%% -*-LaTeX-*-
%%% parallelsignificance.tex.orig
%%% Prettyprinted by texpretty lex version 0.02 [21-May-2001]
%%% on Wed Dec 21 09:24:22 2022
%%% for Steve Dunbar (sdunbar@family-desktop)

\documentclass[12pt]{article}

\input{../../../../etc/macros} %\input{../../../../etc/mzlatex_macros}
\input{../../../../etc/pdf_macros}

\bibliographystyle{plain}

\begin{document}

\myheader \mytitle

\hr

\sectiontitle{Parallel Chain Significance Testing}

\hr

\usefirefox

\hr

% \visual{Study Tip}{../../../../CommonInformation/Lessons/studytip.png}
% \section*{Study Tip}

% \hr

\visual{Rating}{../../../../CommonInformation/Lessons/rating.png}
\section*{Rating} %one of
% Everyone: contains no mathematics.
% Student: contains scenes of mild algebra or calculus that may require guidance.
% Mathematically Mature:  may contain mathematics beyond calculus with
% proofs.
Mathematicians Only:  prolonged scenes of intense rigor.

\hr

\visual{Section Starter Question}{../../../../CommonInformation/Lessons/question_mark.png}
\section*{Section Starter Question} For what positive values of \(
\epsilon \) is \( \epsilon < 2 \epsilon < \sqrt{2 \epsilon} \)?  For
purposes of this section, for what values of \( \epsilon \) is the
``Besag Clifford Test'' better than the ``Parallel \( 2 \epsilon \)
Test'' which in turn is more discriminating than the ``Serial \( \sqrt{2
\epsilon} \) Test?

\hr

\visual{Key Concepts}{../../../../CommonInformation/Lessons/keyconcepts.png}
\section*{Key Concepts}
\renewcommand{\theenumii}{\arabic{enumii}}
\begin{enumerate}
    \item
        \begin{theorem}[Parallel \(2 \epsilon\) Test]
            \label{thm:parallelsignificance:twoepstest}
            \begin{enumerate}
                \item
                    Let \( X \) be a reversible Markov chain with
                    stationary distribution \( \pi \).
                \item
                    Let \( Y_0, Y_1 \dots, Y_n \) and \( Z_0, Z_1, \dots
                    Z_n \) be \( 2 \) independent trajectories of length
                    \( n \) from \( X \) from a common starting point \(
                    Y_0 = Z_0 = \sigma_0 \).
                \item
                    Let \( v :  \mathcal{X} \to \Reals \) be a value
                    function.
                \item
                    Suppose \( \sigma_0 \sim \pi \).
            \end{enumerate}
            Then for any \( n \)
            \begin{multline*}
                \Prob{v(\sigma_0) \text{is an \( \epsilon \)-outlier
                among } \right.  \\
                \left.  v(\sigma_0), v(Y_0), v(Y_1) \dots, v(Y_n), v(Z_1),
                \dots, v(Z_n)} < 2 \epsilon.
            \end{multline*}

        \end{theorem}
    \item
        \begin{theorem}[Besag-Clifford Test]
            \label{thm:parallelsignificance:bc1}
            \begin{enumerate}
                \item
                    Let \( X \) be a reversible Markov chain on \(
                    \mathcal{X} \) with a stationary distribution \( \pi
                    \).
                \item
                    Let \( v :  \mathcal{X} \to \Reals \) be a label
                    function.
                \item
                    Fix a positive integer \( n \).
                \item
                    Suppose \( \sigma_0 \sim \pi \).
                \item
                    Integer \( \xi \) is chosen uniformly in \( \set{0,\dots,
                    n} \).
                \item
                    Let two independent trajectories \( Y_0 , Y_1 \dots \)
                    and \( Z_0, Z_1, \dots \) start from \( Y_0 = Z_0 =
                    \sigma_0 \).
            \end{enumerate}
            Then for any \( n \)
            \begin{multline*}
                \Prob{v(\sigma_0) \text{ is an \( \epsilon \)-outlier
                among } \right.  \\
                \left.  v (\sigma_0), v(Y_1), \dots, v(Y_{\xi} ), v(Z_1),
                \dots, v(Z_{n-\xi} )} \le \epsilon.
            \end{multline*}
        \end{theorem}
    \item
        \begin{theorem}
            \label{thm:parallelsignificance:thm3point1}
            \begin{enumerate}
                \item
                    Let \( X \) be a reversible Markov chain.
                \item
                    Let
                    \begin{align*}
                        \mathbf{X}^1 &= (X_0^1, X_1^1, \dots, X_n^1 )\\
                        &\vdots \\
                        \mathbf{X}^m &= (X_0^m, X_1^m, \dots, X_n^m )\\
                    \end{align*}
                    be \( m \) independent trajectories of length \( n \)
                    from \( X \) starting from a common starting point \(
                    X_0^1 = X_0^2 = \cdots = X_0^m = \sigma_0 \).
                \item
                    Let \( v :  \mathcal{X} \to \Reals \) be a value
                    function.
                \item
                    Define the random variable \( \mathcal{N} \) to be
                    the number of trajectories \( \mathbf{X}^i \) on
                    which \( \sigma_0 \) is an \( \epsilon \)-outlier.
                \item
                    Suppose \( \sigma_0 \) is not an \( (\epsilon,
                    \alpha) \)-outlier.
            \end{enumerate}
            Then
            \[
                \Prob{\mathcal{N} \ge m \sqrt{\frac{2\epsilon}{\alpha}}
                + r } \le \EulerE^{ -r^{2}/ (2 m\sqrt{2\epsilon/\alpha}
                + r)}.
            \]
        \end{theorem}
    \item
        \begin{theorem}
            \label{thm:parallelsignificance:thm6point1}
            \begin{enumerate}
                \item
                    Let \( X \) be a reversible Markov chain.
                \item
                    Let \( n_1, \dots, n_m \) be \( m \) integers
                    independently drawn from a geometric distribution.
                \item
                    Let
                    \begin{align*}
                        \mathbf{X}^1 &= (X_0^1, X_1^1, \dots, X_{n_1}^1
                        )\\
                        &\vdots \\
                        \mathbf{X}^m &= (X_0^m, X_1^m, \dots, X_{n_m}^m
                        )\\
                    \end{align*}
                    be \( m \) independent trajectories, each
                    respectively of length \( n_{i} \), from \( X \)
                    starting from a common starting point \( X_0^1 = X_0^2
                    = \cdots = X_0^m = \sigma_0 \).
                \item
                    Let \( v :  \mathcal{X} \to \Reals \) be a value
                    function.
                \item
                    Define the random variable \( \mathcal{N} \) to be
                    the number of trajectories \( \mathbf{X}^i \) on
                    which \( \sigma_0 \) is an \( \epsilon \)-outlier.
                \item
                    Suppose \( \sigma_0 \) is not an \( (\epsilon,
                    \alpha) \)-outlier.
            \end{enumerate}
            Then
            \[
                \Prob{\mathcal{N} \ge m \sqrt{\frac{\epsilon}{\alpha}} +
                r} \le \EulerE^{r^2/( m \sqrt{\alpha/(\epsilon)} + r)}.
            \]
        \end{theorem}
\end{enumerate}
\renewcommand{\theenumii}{\alph{enumii}}

\hr

\visual{Vocabulary}{../../../../CommonInformation/Lessons/vocabulary.png}
\section*{Vocabulary}
\begin{enumerate}
    \item
        For a fixed value of \( n \), the state \( \sigma_0 \) is an
        \defn{\( (\epsilon, \alpha) \)-outlier} in \( \mathcal{X} \) if,
        among all states in \( \mathcal{X} \), \( \rho(0; \epsilon, n
        \given \sigma_{0}) \) is in the largest \( \alpha \) fraction of
        the values of \( \rho(0; \epsilon, k \given \sigma_{0}) \) over
        all states \( \sigma \in \mathcal{X} \), weighted according to \(
        \pi \).  In particular, being an \( (\epsilon, \alpha) \)-outlier
        measures the likelihood of \( \sigma_0 \) failing the local
        outlier test, ranked against all other states \( \sigma \sim \pi
        \) of the chain \( X \).
\end{enumerate}

\hr

\section*{Notation}
\begin{enumerate}
    \item
        \( n \) -- integer number of steps for a Markov chain
    \item
        \( X \), \( X_0 \), \( X_n \) -- Markov chain, starting state,
        general step of the chain.
    \item
        \( \mathcal{X} \) -- State space of a Markov chain
    \item
        \( v :  \mathcal{X} \to \Reals \) -- value or label function
    \item
        \( \pi \) -- stationary distribution
    \item
        \( x_0 \) -- a given state from the state space
    \item
        \( \sigma_0 \) -- starting state from the stationary
        distribution
    \item
        \( Y_0 , Y_1 \dots \) and \( Z_0, Z_1, \dots \) -- two
        independent trajectories starting from \( Y_0 = Z_0 = \sigma_0 \).

    \item
        \( \alpha_0, \alpha_1, \dots \alpha_n \) -- arbitrary real
        numbers for the definition of outliers
    \item
        \( \xi \) -- Integer chosen uniformly in \( \set{0,\dots, k} \).

    \item
        \( \epsilon \) -- small real number
    \item
        \( \rho(j; \ell, n) \) -- the probability \( X_j \) is among-\(
        \ell \)-smallest from \( X_0, \dots, X_n \)
    \item
        \( \rho(j; \ell, n \given \sigma) \) -- the probability \( X_j =
        \sigma \) is among-\( \ell \)-smallest from \( X_0, \dots, X_n \)
        given the Markov passes through \( \sigma \) at step \( j \)
    \item
        \( \rho(i; \epsilon, n) \) -- the probability that in a \( n \)-step
        stationary trajectory \( X_0, X_1, \dots X_k \), \( v(X_i) \) is
        among the smallest \( \epsilon \)-fraction of indices in the
        list \( v(X_0), v(X_1), \dots, v(X_n) \)
    \item
        \( \rho(0; \epsilon, k \given \sigma_0) \) -- the probability
        that on a trajectory \( \sigma_0 = X_0, X_1, \dots, X_n \), \( v
        (\sigma_0) \) is among the smallest \( \epsilon \)-fraction of
        values in the list \( v(X_0), v(X_1), \dots, v(X_k) \)
    \item
        \( \xi_j \) -- the indicator variable that is \( 1 \) whenever \(
        X_j \) is among-\( \ell \)-smallest among \( X_0, \dots , X_n \)
    \item
        For a fixed value of \( n \), the state \( \sigma_0 \) is an
        \defn{\( (\epsilon, \alpha) \)-outlier} in \( \mathcal{X} \) if,
        among all states in \( \mathcal{X} \), \( \rho(0; \epsilon, n
        \given \sigma_{0}) \) is in the largest \( \alpha \) fraction of
        the values of \( \rho(0; \epsilon, k \given \sigma_{0}) \) over
        all states \( \sigma \in \mathcal{X} \), weighted according to \(
        \pi \).
    \item
        \( m \) -- number of independent trajectories of length \( n \)
        from \( X \) starting from a common starting point \( X_0^1 = X_0^2
        = \cdots = X_0^m = \sigma_0 \).
    \item
        \( \mathcal{N} \) -- the random variable number of trajectories \(
        \mathbf{X}^i \) on which \( \sigma_0 \) is an \( \epsilon \)-outlier.
    \item
        \( \rho(j; \epsilon,k \given \sigma) \) -- the probability that \(
        v(X_j) \) is in the lowest \( \epsilon \) fraction of the values
        \( v(X_0), \dots, v(X_n) \) conditioned on the event \( X_j =
        \sigma \).
    \item
        \( \mu \) -- mean of a geometric distribution
    \item
        \( \mathbf{X} = (X_0^m, X_1^m, \dots, X_k^m) \) -- one of \( m \)
        independent trajectories of length \( n \) from \( X \) starting
        from a common starting point \( X_0^1 = X_0^2 = \cdots = X_0^m =
        \sigma_0 \).
    \item
        \( \rho(0; \epsilon, \mu \given \sigma) \) -- the probability
        that \( v(X_j) \) is in the lowest \( \epsilon \) fraction of
        the values \( v(X_0), \dots, v(X_n) \), conditioned on the event
        \( X_0 = \sigma \), where the length \( n \) is chosen from a
        geometric distribution with mean \( \mu \) supported on \(
        0,1,2,\dots \);
\end{enumerate}
\visual{Mathematical Ideas}{../../../../CommonInformation/Lessons/mathematicalideas.png}
\section*{Mathematical Ideas}

A 1989 paper of Besag and Clifford
\cite{besag89} described a test related to the \( \sqrt{2\epsilon} \)
test % based on Proposition~%\ref{thm:significance:basethm}
to bound the probability that a state from a Markov chain is unusual.
This section derives the Besag-Clifford Test and more sensitive
statistical tests to detect that a given state from a stationary
distribution of a reversible Markov chain is unusual.  The \emph{research
hypothesis} is again:
\begin{quote}
    The given state is unusual with respect to some value measure.
\end{quote}
The \emph{null hypothesis} is again:
\begin{quote}
    The given state is not unusual with respect to some value measure.
\end{quote}

To simplify the interpretation of these tests, the approach in this
subsection will show these tests can efficiently separate the
probability of being unusual from the size of the effect.

\subsection*{Two Chain Significance Tests}

Recall
% from the proof of Proposition~%\ref{thm:serialsignificance:basethm}
the definitions
\[
    \rho(j; \ell, n) = \Prob{v(X_j) \text{ is among- \)\ell\( -smallest
    among } v(X_0), \dots, v(X_n) }
\] and
\[
    \rho(j; \ell, n \given \sigma) = \Prob{v(X_j) \text{ is among- \)\ell\(
    -smallest among } v(X_0), \dots, v(X_n)%
    \given X_0 = \sigma}
\] for \( 0 \le j, \ell \le n \). With a slight change in the parameter
notation, define the \( n+1 \)-vector
\[
    (\rho(0; \epsilon, n), \rho(1; \epsilon, n), \dots, \rho(n; \epsilon,
    n))
\] where for each \( i \), \( \rho(i; \epsilon, n) \) is the probability
that in a \( n \)-step stationary trajectory \( X_0, X_1, \dots X_k \), \(
v(X_i) \) is among the smallest \( \epsilon \)-fraction of indices in
the list \( v(X_0), v(X_1), \dots, v(X_n) \).  Define \( \rho(0;
\epsilon, k \given \sigma_0) \) as the probability that on a trajectory \(
\sigma_0 = X_0, X_1, \dots, X_n \), \( v(\sigma_0) \) is among the
smallest \( \epsilon \)-fraction of values in the list \( v(X_0), v(X_1),
\dots, v(X_k) \).  The goal of the next theorem is to get an improved
bound on these probabilities using parallel Markov chains. First comes a
simple lemma showing that joining two independent, reversible, and
stationary chains still has the stationary distribution.

\begin{lemma}
    \label{lem:parallelsignificance:stationary}
    \begin{enumerate}
        \item
            Let \( X \) be a reversible Markov chain with stationary
            distribution \( \pi \).
        \item
            Let \( Y_0, Y_1 \dots, Y_n \) and \( Z_0, Z_1, \dots Z_n \)
            be \( 2 \) independent trajectories of length \( n \) from \(
            X \) from a common starting point \( Y_0 = Z_0 = \sigma_0 \).
        \item
            Suppose \( \sigma_0 \sim \pi \).
    \end{enumerate}
    Then
    \[
        Y_n, Y_{n-1}, \dots, Y_1, \sigma_0, Z_1, Z_2, \dots, Z_n
    \] is a \( \pi \)-stationary trajectory.
\end{lemma}

\begin{proof}
    \begin{enumerate}
        \item
            Stationarity implies
            \[
                (Z_0, Z_1, \dots, Z_n) \sim (X_n, X_{n+1}, \dots, X_{2n}).
            \]
        \item
            \label{enum:parallelsignificance:lem1step2} Stationarity and
            reversibility imply
            \[
                (Y_n, Y_{n-1}, \dots, Y_1, Y_0) \sim (X_0, X_1, \dots, X_n).
            \]
        \item
            \label{enum:parallelsignificance:lem1step3} The assumption
            that \( Y_1, Y_2, \dots \) and \( Z_1, Z_2, \dots \) are
            independent trajectories from \( \sigma_0 \) is equivalent
            to the condition
            \begin{align*}
                & \Prob{ Z_j = z_j \given Z_{j-1} = z_{j-1}, \dots, Z_1
                = z_1, Z_0 = Y_0=\sigma_0, Y_1=y_1, \dots, Y_n = y_n} \\
                & \qquad = \Prob{ Z_j = z_j \given Z_{j-1} = z_{j-1},
                \dots, Z_1 = z_1, Z_0 = \sigma_0} \\
                & \qquad = \Prob{ Z_j = z_j \given Z_{j-1} = z_{j-1}} \\
                &\qquad = \Prob{ X_{n+j} = z_j \given X_{n+j-1} = z_{j-1}}.
            \end{align*}
        \item
            By induction on \( j \ge 1 \), using step~%
            \ref{enum:parallelsignificance:lem1step2} as the base case
            and step~%
            \ref{enum:parallelsignificance:lem1step3} as the induction
            step,
            \[
                (Y_n, Y_{n-1}, \dots, Y_0=Z_0, Z_1, \dots, Z_j) \sim (X_0,
                X_1, \dots, X_n, X_{n+1}, \dots, X_{n+j}).
            \]
        \item
            Using \( j = n \) and \( Y_0 = Z_0 = \sigma_0 \),
            \[
                (Y_n, Y_{n-1}, \dots, \sigma_0, Z_1, \dots, Z_n) \sim (X_0,
                X_1, \dots, X_n, X_{n+1}, \dots, X_{2n}).
            \]
    \end{enumerate}
\end{proof}

Recall, a real number \( \alpha_0 \) is an \( \epsilon \)-outlier among \(
\alpha_0, \dots \alpha_n \) if
\[
    \card{\setof{\nu}{ \alpha_{\nu} \le \alpha_0 }} \le \epsilon \cdot (n+1).
\]

\begin{theorem}[Parallel \(2 \epsilon\) Test]
    \label{thm:parallelsignificance:twoepstest}
    \begin{enumerate}
        \item
            Let \( X \) be a reversible Markov chain with stationary
            distribution \( \pi \).
        \item
            Let \( Y_0, Y_1 \dots, Y_n \) and \( Z_0, Z_1, \dots Z_n \)
            be \( 2 \) independent trajectories of length \( n \) from \(
            X \) from a common starting point \( Y_0 = Z_0 = \sigma_0 \).
        \item
            Let \( v :  \mathcal{X} \to \Reals \) be a value function.
        \item
            Suppose \( \sigma_0 \sim \pi \).
    \end{enumerate}
    Then for any \( n \)
    \begin{multline*}
        \Prob{v(\sigma_0) \text{is an \( \epsilon \)-outlier among } v(\sigma_0),
        v(Y_0), v(Y_1) \dots, v(Y_n), v(Z_1), \dots, v(Z_n)} \\
        < 2\epsilon.
    \end{multline*}
\end{theorem}
\index{Parallel \(2 \epsilon\) Test}

\begin{proof}
    \begin{enumerate}
        \item
            Given a Markov chain \( X \) with values \( v:  \mathcal{X}
            \to \Reals \) and stationary distribution \( \pi \), define
            for each \( j, \ell \le n \) the probability \( \rho(j; \ell,
            n) \) that for a \( \pi \)-stationary trajectory \( X_1, X_2,
            \dots, X_n \) the value \( v(X_j) \) is among-\( \ell \)-smallest
            among \( v(X_1), v(X_2), \dots, v(X_n) \).  All \( \pi \)-stationary
            trajectories \( X_1, X_2, \dots, X_n \) of fixed length \( n
            \) are all identical in distribution so \( \rho(j; \ell, n) \)
            is well-defined.  (See the exercises in the previous
            section.)
        \item
            If the sequence \( X_1, X_2, \dots X_n \) is a \( \pi \)-stationary
            trajectory for \( X \), then \( (X_{n-j}, \dots X_n, \dots,
            X_{2n-j}) \) is an identically distributed sequence.  Thus
            the probability that \( v(X_n) \) is among-\( \ell \)-smallest
            among
            \[
                v(X_{n-j}), \dots, v(X_n), \dots v(X_{2n-j})
            \] is \( \rho(j; \ell, n) \).
        \item
            \label{enum:parallelsignifcance:step2} Observe that the
            event
            \[
                [v(X_n) \text{ is among- \)\ell\( -smallest among } v(X_
                {n-j}), \dots, v(X_n), \dots, v(X_{2n-j})]
            \] occurs if the event
            \[
                [v(X_n) \text{ is among- \)\ell\( -smallest among } v(X_0),
                \dots, v(X_n), \dots, v(X_{2n})]
            \] occurs, for all \( j = 0, \dots n \).  It follows that \(
            \rho(n; \ell, 2n) \le \rho(j; \ell, n) \).
        \item
            Let \( \xi_j \) be the indicator variable that is \( 1 \)
            whenever \( X_j \) is among-\( \ell \)-smallest among \( X_0,
            \dots , X_n \) The sum \( \sum_{\nu=0}^n \rho( \nu; \ell, n)
            = \E{\sum_{\nu=0}^n \xi_{\nu}} \) is the expected number of
            indices \( j \in \set{0, \dots, n} \) such that \( v(X_j) \)
            is among-\( \ell \)-smallest from \( v(X_0), \dots, v(X_n) \),
            so \( \sum_{\nu=0}^n \rho (\nu; \ell, n) \le \ell + 1 \).
        \item
            \label{enum:parallelsignifcance:step4} By~%
            \ref{enum:parallelsignifcance:step2},
            \[
                (n+1) \rho(n; \ell, 2n) \le \sum_{\nu=0}^n \rho(\nu;
                \ell, n).
            \] Further
            \begin{equation}
                \label{eq:parallelsignificance:eqn1} \sum_{\nu=0}^n \rho
                (\nu; \ell, n) \le \ell + 1.
            \end{equation}
            (The proof of the next theorem also uses this inequality.)
        \item
            Then
            \[
                \rho(n; \ell, 2n) = \frac{1}{n+1} \sum_{\nu=0}^n \rho(\nu;
                \ell, n) \le \frac{\ell + 1}{n+1} < 2 \cdot \frac{\ell+1}
                {2n+1}.
            \]
        \item
            As in the section on Serial Significance, use the slight
            abuse of notation that \( \rho(i; \epsilon, n) \) is the
            probability that in a \( n \)-step stationary trajectory \(
            X_0, X_1, \dots X_n \), \( v(X_i) \) is among-\( \lfloor
            \epsilon(n+1) \rfloor \)-smallest in the list \( v(X_0), v(X_1),
            \dots, v(X_n) \).
        \item
            Then
            \[
                \rho(n; \epsilon, 2n) \le \ 2 \frac{ \lfloor \epsilon (2n+1)
                \rfloor}{2n+1} \le 2 \frac{ \epsilon (2n+1)}{2n+1} = 2\epsilon.
            \]
    \end{enumerate}
\end{proof}

\begin{remark}
    The remaining theoretical question is what is the largest possible
    value of \( \rho(n; \epsilon, 2n) \) as a function of \( \epsilon \).
\end{remark}

\begin{remark}
    Using Markov chains is an inherently serial process, each step
    follows from the previous step.  However, as the name suggests,
    Theorem~%
    \ref{thm:parallelsignificance:twoepstest} creates the opportunity to
    use parallel computing to increase the ability to detect outliers.
    Later Theorems will expand testing to multiple chains.
\end{remark}

\begin{example}
    Consider a specific simple case of the alternative Ehrenfest urn
    model.%
    \index{Ehrenfest urn model}
    The setup of the urn model is repeated here again for convenient
    reference. Two urns, labeled \( A \) and \( B \), contain a total of
    \( N = 7 \) balls.  At each step a ball is selected at random with
    all selections equally likely.  Then an urn is selected, urn \( A \)
    with probability \( p = \frac{1}{2} \) and urn \( B \) with
    probability \( q = 1-p = \frac {1}{2} \) and the ball is moved to
    that urn.  The state at each step is the number of balls in the urn \(
    A \), from \( 0 \) to \( N \).  Start from state \( 0 \), with no
    balls in urn \( A \).  All states are accessible and all states
    communicate, so the chain is irreducible.  All states are recurrent
    with no transient states.  The chain is periodic with period \( 1 \)
    so it is regular.  The standard Ehrenfest urn model, without the
    random choice of urns at each step, is periodic with period \( 2 \).
    Having period \( 1 \) makes interpretation of the results simpler.

    For this simple example, let the value function be the number of
    balls in urn \( A \), so that the value function is identical with
    the state number.  The simplicity of the alternative Ehrenfest urn
    model limits the possibilities for value functions.

    This example uses the reproducible simulation of the alternate
    Ehrenfest urn model in \texttt{simParRhoProb.R}. The path length, or
    number of steps, is \( 10 \), short enough to be comprehensible,
    long enough have some content.  Including the start state, the
    Markov chain has \( 11 \) values. The number of trials is \( 15 \),
    enough to illustrate the variety of sample paths, but not too many
    to overwhelm.

    For each trial, first the simulation generates a chain of length \(
    11 \) and the values along the paths are recorded in the rows of an \(
    15 \times 11 \) matrix. Then for each given step \( j \) along the
    path, the count of other step values along the path less than or
    equal to the given step value is compared to \( \ell \).  If the
    count is less than or equal to \( \ell \), that makes the given step
    value at \( j \) among-\( \ell \)-smallest and the count in a \( 11
    \times 11 \) matrix is incremented.  When done, divide the counting
    matrix by the number of trials, \( 15 \), to give an empirical
    estimate of \( \rho(j; 0 ,n) \).  This is just as in the example in
    the section on Serial Significance.  Summing down each column gives
    the values in Table~%
    \ref{tab:parallelsignificance:sumofrhoprobs}. The table illustrates
    the inequality \( \sum_{\nu=0}^n \rho (\nu; \ell, n) \le \ell + 1 \)
    in step~%
    \ref{enum:parallelsignifcance:step4} in the proof of the Parallel \(
    2 \epsilon \) Test.

    \begin{table}
        \centering
        \begin{tabular}{c|ccccccccccc}
            \( \ell +1 \)                           & 1     & 2     & 3     & 4     & 5     & 6     & 7     & 8     & 9     & 10    & 11     \\ 
            \hline
            \( \sum_{\nu=0}^n \rho(\nu; \ell, n) \) & 0.267 & 0.667 & 1.733 & 2.600 & 3.067 & 3.467 & 4.133 & 4.933 & 7.933 & 8.933 & 11.000 \\ 
        \end{tabular}
        \caption{Empirical values of to illustrate the inequality in the
        proof of }%
        \label{tab:parallelsignificance:sumofrhoprobs}
    \end{table}

    For each trial, next the simulation generates two independent chains
    of length \( 11 \) starting from the first state.  One chain with
    the common starting value is reversed and concatenated with the
    second chain and the values along the paths are recorded in the rows
    of an \( 15 \times 21 \) matrix.  This creates trials for paths as
    in Lemma~%
    \ref{lem:parallelsignificance:stationary}. Then for starting step \(
    11 \) value along the concatenated path, the count of other step
    values along the path less than or equal to the starting step value
    is compared to \( \ell \).  If the count is less than or equal to \(
    \ell \), that makes the given step value at \( 11 \) among-\( \ell \)-smallest
    and the count in an \( 11 \) vector is incremented.  When done,
    divide the vector by the number of trials, \( 15 \), to give an
    empirical estimate of \( \rho(n; \ell , 2n) \) as in step~%
    \ref{enum:parallelsignifcance:step2} of the proof. Comparing the
    empirical probabilities illustrates \( \rho(n; \ell, 2n) \le \rho(j;
    \ell, n) \) with this simulation.

    Finally, for \( \epsilon = 0.1 \), the empirical probability of the
    starting value being an \( \epsilon \)-outlier among the values form
    the concatenated path is the empirical probability \( \rho(10, 2, 20)
    = 0 \) which is less than \( \epsilon \).
\end{example}

\begin{theorem}[Besag-Clifford Test]
    \label{thm:parallelsignificance:bc1}
    \begin{enumerate}
        \item
            Let \( X \) be a reversible Markov chain on \( \mathcal{X} \)
            with a stationary distribution \( \pi \).
        \item
            Let \( v :  \mathcal{X} \to \Reals \) be a label function.
        \item
            Fix a positive integer \( n \).
        \item
            Suppose \( \sigma_0 \sim \pi \).
        \item
            Integer \( \xi \) is chosen at random uniformly in \( \set{0,\dots,
            n} \).
        \item
            Let two independent trajectories \( Y_0 , Y_1 \dots \) and \(
            Z_0, Z_1, \dots \) start from \( Y_0 = Z_0 = \sigma_0 \).
    \end{enumerate}
    Then for any \( n \)
    \begin{multline*}
        \Prob{(v(\sigma_0)) \text{ is an \( \epsilon \)-outlier among }
        \right.  \\
        \left.  v(\sigma_0), v(Y_1), \dots, v(Y_{\xi} ), v(Z_1), \dots,
        v(Z_{n-\xi} )} \le \epsilon.
    \end{multline*}
\end{theorem}
\index{Besag-Clifford Test}

\begin{remark}
    See Figure~%
    \ref{fig:parallelsignificance:BCtest} for a schematic diagram of the
    Besag-Clifford Test.  Note that \( \sigma_0, Y_1, \dots, Y_{\xi}, Z_1,
    \dots, Z_{n-\xi} \) may not be a valid Markov chain.  For instance
    the transition \( Y_{\xi} \) to \( Z_1 \) may have probability \( 0 \).
    Since the Theorem only considers the probability of a set of values,
    failure to be a valid Markov chain sequence is not a problem.
\end{remark}

% The following figure is drawn with the primitive LaTeX picture
% environment rather than the more sophisticated Asymptote, since it
% is mostly math environments.  All positions, lengths, and parameters
% are established by experimentation,  I should probably use \shortparallel
% instead of \parallel, but it seemed to not be available.
\begin{figure}
    \centering
    \begin{picture}(100,60)(0,-25)
        \put(0,30){\( Y_0 \rightarrow Y_1 \rightarrow \cdots \rightarrow
        Y_{\xi} \rightarrow \cdots \rightarrow Y_n \)}
        \put(0,15){\( \parallel \)}
        \put(0,0){\( \sigma_0 \)}
        \put(0,-15){\( \parallel \)}
        \put(0,-30){\( Z_0 \rightarrow Z_1 \rightarrow \cdots
        \rightarrow Z_{n - \xi} \rightarrow \cdots \rightarrow Z_n \)}
        \put(90,27){\vector(-1,-1){45}}
    \end{picture}
    \caption{Schematic diagram of the Besag-Clifford Test.  The short
    wide-head arrows are the Markov chain transitions, the long
    narrow-head arrow represents the switch from the \( Y \) trajectory
    to the \( Z \) trajectory.}%
    \label{fig:parallelsignificance:BCtest}
\end{figure}

\begin{proof}[Proof of Theorem~\ref{thm:parallelsignificance:bc1}]
    \begin{enumerate}
        \item
            The proof uses only equation~%
            \eqref{eq:parallelsignificance:eqn1}.
        \item
            Recall from the proof of Theorem~%
            \ref{thm:parallelsignificance:twoepstest} that the \( \rho(j;
            \ell, n) \) are fixed real numbers associated to a
            stationary Markov chain.
        \item
            If \( \ell \) and \( n \) are fixed, and \( \xi \) is chosen
            at random uniformly from \( \set{0, \dots, n} \), then the
            resulting \( \rho(\xi; \ell, n) \) is a random variable
            uniformly distributed on the set of real numbers \( \set{\rho
            (0; \ell, n), \dots, \rho(n; \ell, n)} \).
        \item
            The probability that \( v(\sigma_0) \) is among-\( \ell \)-smallest
            from
            \[
                v(\sigma_0), v(Y_1), \dots, v(Y_{\xi}), v(Z_1), \dots, v
                (Z_ {n-\xi})
            \] is
            \[
                \frac{1}{n+1} (\rho(0; \ell, n)+ \rho(1; \ell, n) +
                \cdots + \rho(n; \ell, n) ) \le \frac{\ell + 1}{n + 1}
            \] using equation~\eqref{eq:parallelsignificance:eqn1}.
        \item
            This observation uses a variant of Lemma~%
            \ref{lem:parallelsignificance:stationary} that for any \( j \),
            \[
                Y_j, \dots, Y_1, \sigma_0, Z_1, \dots, Z_{n-j}
            \] is a \( \pi \)-stationary trajectory.
    \end{enumerate}
\end{proof}

%%%%% \begin{remark}
%%%%%     An intuitive interpretation of the theorem is that typical, i.e.\ %
%%%%%     stationary, states are unlikely to change, as measured by \( v \),
%%%%%     in a consistent way under two sequences of chain transitions of
%%%%%     random complementary lengths. [Is there a better way to state this?]
%%%%% \end{remark}

\begin{remark}
    Notice that \( \epsilon \) would be the correct value of the
    probability if, for example, the Markov chain is simply a collection
    of independent random uniform samples from \( [0,1] \) with the
    identity as the value function.  The values of the \( Y \) and \( Z \)
    trajectories would then be a collection of \( 2n + 1 \) random
    uniform samples from \( [0,1] \) and the probability of \( \Prob{v(\sigma_0)}
    \) is an \( \epsilon \)-outlier would be \( \epsilon \).

    The striking thing about Theorem~%
    \ref{thm:parallelsignificance:bc1} is that it achieves a
    best-possible dependence on the parameter \( \epsilon \).  The
    sacrifice is in the theorem's more complicated intuitive
    interpretation.

    In applications of these statistical tests to aspects of public
    policy such as the gerrymandering example in~%
    \cite{doi:10.1080/2330443X.2020.1806763}, it is desirable to have
    tests with simple, intuitive interpretations.  To enable better
    significance testing in the policy domain, one goal of this section
    is to prove a theorem enabling Markov chain significance testing
    which is intuitively interpretable in the sense of Proposition~%
    \ref{thm:serialsignificance:basethm} while having linear dependence
    on \( \epsilon \), as in Theorem~%
    \ref{thm:parallelsignificance:bc1}.
\end{remark}

\begin{remark}
    One common feature of the tests based on Proposition~%
    \ref{thm:significance:basethm} and~%
    \ref{thm:parallelsignificance:bc1} is the use of randomness.  In
    particular, the probability space underneath these theorems includes
    both the random choice of \( \sigma_0 \) assumed by the null
    hypothesis and the random steps taken by the Markov chain from \(
    \sigma_0 \).  Thus the measures of ``how much (globally) unusual'' \(
    \sigma_0 \) is with respect to its performance in the local outlier
    test and ``how sure'' we are that \( \sigma_0 \) is unusual in this
    respect are intertwined in the final \( p \)-value.  In particular,
    the effect size and the statistical significance are not explicitly
    separated.
\end{remark}

\begin{example}
    Once again, consider a specific simple case of the alternative
    Ehrenfest urn model with \( N = 7 \) balls.%
    \index{Ehrenfest urn model}
    This example uses the reproducible simulation of the alternate
    Ehrenfest urn model in \texttt{simBesagClifford.R}. The path length,
    or number of steps, is \( 10 \), short enough to be comprehensible,
    long enough have some content.  Including the start state, the
    Markov chain has \( 11 \) values. The number of trials is \( 15 \),
    enough to illustrate the variety of sample paths, but not too many
    to overwhelm.

    For each trial, the simulation generates two independent chains of
    length \( 11 \) starting from the common starting state.  Choosing
    integer \( \xi \) at random uniformly in \( \set{0, \dots, n} \) the
    values \( v(\sigma_0), v(Y_1), \dots, v(Y_{\xi} ), v(Z_1), \dots, v(Z_
    {n-\xi}) \) are recorded in the rows of an \( 15 \times 11 \)
    matrix.  This creates trials as in Theorem~%
    \ref{thm:parallelsignificance:bc1}. Then for starting step \( 11 \)
    value along the concatenated path, the count of other step values
    along the path less than or equal to the starting step value is
    compared to \( \ell \).  If the count is less than or equal to \(
    \ell \), that makes the given step value at \( 11 \) among-\( \ell \)-smallest
    and the count in an \( 11 \) vector is incremented.  When done,
    divide the vector by the number of trials, \( 15 \), to give an
    empirical estimate of the probability of an outlier as in Theorem~%
    \ref{thm:parallelsignificance:bc1}. Comparing the empirical
    probabilities illustrates the Besag-Clifford Test with this
    simulation.

    Finally, for \( \epsilon = 0.1 \), the empirical probability of the
    starting value being an \( \epsilon \)-outlier among the values form
    the concatenated path is the empirical probability of being among-\(
    1 \)-smallest, namely \( 0 \) which is less than \( \epsilon \).

\end{example}
\subsection*{The Multiple Chain Significance Test}

Extend the slight abuse of notation adopted in the previous section. For
each \( i \), let \( \rho(j; \epsilon, n \given \sigma) \) be the
probability that in an \( n \)-step stationary trajectory \( X_0, X_1,
\dots X_n \), \( v(X_i) \) is among-\( \lfloor \epsilon(n+1) \rfloor \)-smallest
in the list \( v(X_0), v(X_1), \dots, v(X_n) \) given that \( X_j =
\sigma \).

\begin{definition}
    For a fixed value of \( n \), the state \( \sigma_0 \) is an \defn{\(
    (\epsilon, \alpha) \)-outlier} in \( \mathcal{X} \) if, among all
    states in \( \mathcal{X} \), \( \rho(0; \epsilon, n \given \sigma_{0})
    \) is in the largest \( \alpha \) fraction of the values of \( \rho(0;
    \epsilon, n \given \sigma_{0}) \) over all states \( \sigma \in
    \mathcal{X} \), weighted according to \( \pi \).  In particular,
    being an \( (\epsilon, \alpha) \)-outlier measures the likelihood of
    \( \sigma_0 \) failing the local outlier test, ranked against all
    other states \( \sigma \sim \pi \) of the chain \( X \).
\end{definition}

Whether \( \sigma_0 \) is an \( (\epsilon, \alpha) \)-outlier is a
deterministic question about the properties of \( \sigma_0 \), \( X \),
and \( v \).  It is a deterministic measure (defined in terms of the
Markov chain distribution) of the extent to which \( \sigma_0 \) could
be unusual in value compared to all of \( \mathcal{X} \), keeping in
mind the stationary distribution chain as starting value.

\begin{example}
    Yet once again, consider a specific simple case of the alternative
    Ehrenfest urn model with \( N = 7 \) balls.%
    \index{Ehrenfest urn model}
    This example uses the reproducible simulation of the alternate
    Ehrenfest urn model in \texttt{simEpsAlpha.R}. The path length, or
    number of steps, is \( n = 10 \), short enough to be comprehensible,
    long enough have a little content.  This is not long enough to
    converge to the stationary distribution, so the example is is
    explanatory reasons only.  Including the start state, each
    simulation of the Markov chain has \( 11 \) values. For the example
    take \( \epsilon = 0.1 \), so the smallest fraction of values will
    be \( \lfloor \epsilon (n+1) \rfloor = \lfloor 0.1 \cdot 11 \rfloor
    = 1 \) value.  That is, the value under consideration must be the
    unique minimum of the set of chain values. For this example, in turn use each
    possible state as the starting state. The number of trials for each
    start state is \( 15 \), enough to illustrate the variety of sample
    paths, but not too many to overwhelm. Simulate the Markov chain from
    the start state for each trial and record the path.  The record of
    paths will be a \( 3 \)-dimensional array of size \( 15 \times
    11 \times 8 \).  The array has \( 8 \) planes, one for each start
    state.  Each plane has \( 15 \) rows for the trials, and the \( 11 \)
    columns are the step number, recording the state at that step for
    that trial.  For each trial tally the number of states, aside from
    the start state, which are less than or equal to the start state.
    If that tally is less than or equal to \( 1 \), the start state is
    the unique minimum, and increment the count of ``start state as
    unique minimum'' from the trials. Dividing by the number of trials
    gives an approximation for the empirical probability (from the few
    trials in the simulation) of being an \( \epsilon \)-outlier \emph{starting
    from that state.} According to the definition, the probability
    should be weighted according to \( \pi \).  Table~%
    \ref{tab:parallelsignificance:epsalpha} gives the approximate
    probability of being an \( \epsilon \)-outlier
    starting from the state and the probability of being an
    \(epsilon \)-outlier weighted by the stationary distribution.
    \begin{table}
        \centering
        \begin{tabular}{l|cccccccc}
            State    & 0     & 1     & 2     & 3     & 4     & 5     & 6     & 7     \\ 
                     & 0.467 & 0.333 & 0.267 & 0.067 & 0.000 & 0.000 & 0.000 & 0.000 \\ 
            \( \pi \)& 0.008 & 0.055 & 0.164 & 0.273 & 0.273 & 0.164 & 0.055 & 0.008 \\ 
                     & 0.004 & 0.018 & 0.044 & 0.018 & 0.000 & 0.000 & 0.000 & 0.000 \\ 
        \end{tabular}
        \caption{Empirical estimates of being an \( (\epsilon, \alpha) \)
        outlier for the small alternate Ehrenfest urn model.  The second
        row is the rough probability of being a \( \epsilon \)-outlier
        starting from the state in the top row rounded to three places.
        The third row is the stable distribution \(
        \operatorname{Bin}
        (7, 0.5) \) rounded to three places.  The fourth row is the
        weighted approximation of being an \( \epsilon \)-outlier.  }%
        \label{tab:parallelsignificance:epsalpha}
    \end{table}

    The rough fraction of chains starting from a random state from the
    stationary distribution and resulting in the starting state being an
    \( \epsilon \)-outlier is \( 0.004 + 0.018 + 0.044 + 0.018 = 0.084 \).
    The rough fraction of chains starting from a random state from the
    stationary distribution other than \( 0 \) and resulting in the
    starting state being an \( \epsilon \)-outlier is \( 0.018 + 0.044 +
    0.018 = 0.080 \).  This means that starting from a random state from
    the stationary distribution,
    \( 0 \) is more likely than all but about a \( 0.080 \)-fraction of
    states to have an \( v \)-value in the least \( 0.1 \)-fraction of
    values.  Therefore, from this small and rough but comprehensible
    example, \( 0 \) is an \( (0.1, 0.08) \)-outlier.

\end{example}

\begin{example}
    For another example, fix \( n = 10^9 \).  If \( \sigma_0 \) is a \(
    (10^{-6}, 10^{-5}) \)-outlier for \( X \) and \( \pi \) is the
    uniform distribution, this means that among all states \( \sigma \in
    \mathcal{X} \), \( \sigma_0 \) is more likely than all but a \( 10^{-5}
    \)-fraction of states to have an \( v \)-value in the least \( 10^{-6}
    \)-fraction of values \( v(X_0), v(X_1 ), \dots , v(X_{10^9} ) \).
    The probability space underlying the ``more likely'' claim here just
    concerns the choice of the random trajectory \( X_1, \dots, X_{10^9}
    \) from \( X \).
\end{example}

% The following theorem enables asserting statistical significance for the
% property of being an \( (\epsilon, \alpha) \)-outlier.  In particular,
While tests based on Proposition~%
\ref{thm:serialsignificance:basethm} and Theorem~%
\ref{thm:serialsignificance:powerthm} take as their null hypothesis that
\( \sigma_0 \sim \pi \) is not unusual in the value function, the
following Theorem~%
\ref{thm:parallelsignificance:thm3point1} takes as its null hypothesis \(
\sigma_{0} \) is not an \( (\epsilon, \alpha) \)-outlier.

Recall \( \rho(j; \epsilon, n \given \sigma) \) is the probability that \(
v(X_j) \) is in the lowest \( \epsilon \) fraction of the values \( v(X_0),
\dots, v(X_n) \) conditioned on the event \( X_j = \sigma \).

\begin{lemma}
    If \( \sigma_0 \) is not an \( (\epsilon,\alpha) \)-outlier, then
    \[
        \label{eq:parallelsignificance:nine} \rho(0; \epsilon, n \given
        \sigma_0) \le \sqrt{\frac{2 \epsilon}{\alpha}}.
    \]
\end{lemma}

\begin{proof}
    \begin{enumerate}
        \item
            Consider a \( \pi \)-stationary trajectory \( X_0, \dots, X_n,
            \dots, X_{2n} \) and condition on the event \( X_n = \sigma \)
            for some arbitrary \( \sigma \in \mathcal{X} \).  Since \( X
            \) is reversible, we can view this trajectory as two
            independent trajectories, \( X_n, X_{n+1}, \dots, X_{2n} \)
            and \( X_n, X_{n-1}, X_ {n-2}, \dots, X_0 \) both beginning
            from \( \sigma \).
        \item
            Let \( A \) and \( B \) be the events \( v(X_n) \) is an \(
            \epsilon \)-outlier among the lists \( v(X_0), \dots, v(X_n)
            \) and \( v(X_n), \dots, v(X_{2n}) \), respectively.  The
            event \( A \intersect B \) would have \( v(X_n) \) among the
            lowest \( \epsilon \) fraction of values in the long chain
            of \( 2n \) steps.  By the independence of the trajectories,
            \begin{equation}
                \label{eq:parallelsignificance:eleven} (\rho(0; \epsilon,
                n \given \sigma))^2 = \Prob{ A \intersect B} \le \rho(n;
                \epsilon, 2n \given \sigma).
            \end{equation}
        \item
            The assumption that the given \( \sigma_0 \in \mathcal{X} \)
            is not an \( (\epsilon, \alpha) \)-outlier means that for a
            random \( \sigma \sim \pi \)
            \begin{equation}
                \label{eq:parallelsignificance:twelve} \Prob{ \rho(0;
                \epsilon, n \given \sigma) \ge \rho(0; \epsilon, n
                \given \sigma_0) } \ge \alpha.
            \end{equation}
        \item
            Equation~\eqref{eq:parallelsignificance:eleven} gives \( (\rho
            (0; \epsilon, n \given \sigma))^2 \le \rho(n, \epsilon, 2n
            \given \sigma) \) while Theorem~%
            \ref{thm:parallelsignificance:twoepstest} gives \( \rho(n;
            \epsilon, 2n \given \sigma) \le 2\epsilon \).
        \item
            Taking expectations with respect to a random \( \sigma \sim
            \pi \) gives
            \begin{align*}
                \Esub{\sigma \sim \pi}{(\rho(0; \epsilon, n \given
                \sigma))^2}     & \le \Esub{\sigma \sim \pi}{\rho(n;
                \epsilon, 2n \given \sigma) } \\
                &= \rho(n; \epsilon, 2n) \le 2\epsilon.
            \end{align*}
        \item
            On the other hand, with~\eqref{eq:parallelsignificance:twelve}
            \[
                \Esub{\sigma \sim \pi}{(\rho(0; \epsilon, n \given
                \sigma))^2 } \ge \alpha \cdot (\rho(0; \epsilon, n
                \given \sigma))^2
            \] so that
            \[
                (\rho(0; \epsilon, n \given \sigma))^2 \le \frac{2\epsilon}
                {\alpha}.
            \]
    \end{enumerate}
\end{proof}

\begin{theorem}
    \label{thm:parallelsignificance:thm3point1}
    \begin{enumerate}
        \item
            Let \( X \) be a reversible Markov chain with stationary
            distribution \( \pi \).
        \item
            Let
            \begin{align*}
                \mathbf{X}^1    &= (X_0^1, X_1^1, \dots, X_n^1 )\\
                &\vdots \\
                \mathbf{X}^m    &= (X_0^m, X_1^m, \dots, X_n^m )\\
            \end{align*}
            be \( m \) independent trajectories of length \( n \) from \(
            X \) starting from a common starting point \( X_0^1 = X_0^2
            = \cdots = X_0^m = \sigma_0 \).
        \item
            Let \( v :  \mathcal{X} \to \Reals \) be a value function.
        \item
            Define the random variable \( \mathcal{N} \) to be the
            number of trajectories \( \mathbf{X}^i \) on which \( \sigma_0
            \) is an \( \epsilon \)-outlier.
        \item
            Suppose \( \sigma_0 \) is not an \( (\epsilon, \alpha) \)-outlier.
    \end{enumerate}
    Then
    \[
        \Prob{\mathcal{N} \ge m \sqrt{\frac{2\epsilon}{\alpha}} + r }
        \le \EulerE^{ -r^{2}/ (2 m\sqrt{2\epsilon/\alpha} + r)}.
    \]
\end{theorem}

\begin{proof}
    \begin{enumerate}
        \item
            For a \( \pi \)-stationary trajectory the probability \(
            \rho(0; \epsilon, n \given \sigma_0) \) is the probability
            that on a trajectory \( \sigma_0 = X_0, X_1, \dots, X_n \), \(
            v(\sigma_0) \) is among the smallest \( \epsilon \) fraction
            of values in the list \( v(X_0), v(X_1), \dots, v(X_n) \).
        \item
            The random variable \( \mathcal{N} \) is the number of
            trajectories \( X^i \) from \( \sigma_0 \) on which \(
            \sigma_0 \) is observed to be an \( \epsilon \)-outlier with
            respect to the valuing \( v \).  Using the Lemma, the random
            variable \( \mathcal{N} \) is the sum of \( m \) independent
            Bernoulli random variables each of which takes value \( 1 \)
            with probability less than \( \sqrt{\frac{2 \epsilon}{\alpha}}
            \).
        \item
            Then \( \E{\mathcal{N}} \le m \sqrt{\frac{2 \epsilon}{\alpha}}
            \). By Chernoff's bound (in the form
            \[
                \Prob{X \ge (1 + \delta) \mu } \le \EulerE^{-\delta^2\mu/
                (2 + \delta)}
            \] for a sum of Bernoulli random variables with mean \( \mu \))
            % Wilkipedia Chernoff_bound, section on Multiplicative bound
            \[
                \Prob{\mathcal{N} \ge (1 + \delta) m \sqrt{\frac{2\epsilon}
                {\alpha}} } \le \EulerE^{ -(\delta^{2} m\sqrt{2\epsilon/\alpha})/
                (2 + \delta)}.
            \]
        \item
            Using \( \delta = r/(m \sqrt{2\epsilon/\alpha}) \) and
            substituting gives
            \[
                \Prob{\mathcal{N} \ge m \sqrt{\frac{2\epsilon}{\alpha} +
                r} } \le \EulerE^{ -r^{2}/ (2 m\sqrt{2\epsilon/\alpha} +
                r)}.
            \]
    \end{enumerate}
\end{proof}

% \begin{remark}
%     Apart from separating measures of statistical significance from the
%     quantification of a local outlier, Theorem~%
%     \ref{thm:parallelsignificance:thm3point1} connects the intuitive
%     Local Outlier Test tied to Proposition~%
%     \ref{thm:significance:basethm}, which motivates the definition of \(
%     (\epsilon, \alpha) \)-outlier, to the better quantitative dependence
%     on \( \epsilon \) in Theorem~%
%     \ref{thm:parallelsignificance:bc1} [Not sure what this is trying to
%     say.]
% \end{remark}

\begin{remark}
    When \( m \) is a reasonable size for calculation, it may make more
    sense to use the exact binomial tail in place of the Chernoff bound.
    That is the content of the following Corollaries.
\end{remark}

\begin{corollary}
    Given the hypotheses in Theorem~%
    \ref{thm:parallelsignificance:thm3point1}, then
    \[
        \Prob{\mathcal{N} \ge K} \le \sum\limits_{\nu=K}^m \binom{m}{\nu}
        \left( \sqrt{\frac{2\epsilon}{\alpha}} \right)^{\nu} \left( 1 -
        \sqrt{\frac{2\epsilon}{\alpha}} \right)^{m-\nu}.
    \]
\end{corollary}

\begin{corollary}
    \label{cor:parallelsignificance:cor32} Given the hypotheses of
    Theorem~%
    \ref{thm:parallelsignificance:thm3point1}, then
    \[
        \Prob{\sigma_0 \text{ is an \)\epsilon\( -outlier on all of }
        \mathbf {X}^1, \dots \mathbf{X}^m } \le \left( \frac{2\epsilon}{\alpha}
        \right)^{m/2}.
    \]
\end{corollary}

% \begin{example}
%     To compare the quantitative performance of Theorem~%
%     \ref{thm:parallelsignificance:thm3point1} to Proposition~%
%     \ref{thm:significance:basethm} and Theorem~%
%     \ref{thm:parallelsignificance:bc1}, consider the case of a state \(
%     \sigma_0 \) for which a random trajectory \( \sigma_0 = X_0 , X_1 ,
%     \dots X_n \) is likely (say with some constant probability \( p' \))
%     to find \( \sigma_0 \) an \( \epsilon' \)-outlier.  For Proposition~%
%     \ref{thm:significance:basethm}, significance at \( p \approx \sqrt{2\epsilon}
%     \) would be obtained, while using Theorem~%
%     \ref{thm:parallelsignificance:bc1}, one would hope to obtain
%     a \( p \)-value approximately \( O(\epsilon') \).  Applying Theorem~%
%     \ref{thm:parallelsignificance:thm3point1}, one expects to see \( p \)
%     around \( m p' \).  In particular, one could demonstrate that \(
%     \sigma_0 \) is an \( (\epsilon', \alpha) \)-outlier for \( \alpha =
%     \frac{3\epsilon} {(p')^2} \) (a linear relation on \( \epsilon \))
%     at a \( p \)-value which can be made arbitrarily small (at an
%     exponential rate) by increasing the number of observed trajectories \(
%     m \).  [This needs more explanation.]
% \end{example}

\subsection*{The Improved Multiple Chain Significance Test}

The following Theorem~%
\ref{thm:parallelsignificance:thm6point1} is the analog of Theorem~%
\ref{thm:parallelsignificance:thm3point1} using an analog of Besag and
Clifford's Theorem~%
\ref{thm:parallelsignificance:bc1} in place of Proposition~%
\ref{thm:significance:basethm} in the proof.  This version pays the
price of using a random \( n \) instead of a fixed \( n \) for the
notion of an \( (\epsilon, \alpha) \)-outlier, but has the advantage
that the constant \( 2 \) is eliminated from the bound.  As in Theorem~%
\ref{thm:parallelsignificance:thm3point1}, the notion of \( (\epsilon,
\alpha) \)-outlier here is still just defined with respect to a single
path, although Theorem~%
\ref{thm:parallelsignificance:bc1} depends on using two independent
trajectories.

\begin{theorem}
    \label{thm:parallelsignificance:thm6point1}
    \begin{enumerate}
        \item
            Let \( X \) be a reversible Markov chain.
        \item
            Let \( n_1, \dots, n_m \) be \( m \) integers independently
            drawn from a geometric distribution with mean \( \mu \)
            supported on \( 0,1,2,\dots \).
        \item
            Let
            \begin{align*}
                \mathbf{X}^1    &= (X_0^1, X_1^1, \dots, X_{n_1}^1 )\\
                &\vdots \\
                \mathbf{X}^m    &= (X_0^m, X_1^m, \dots, X_{n_m}^m )\\
            \end{align*}
            be \( m \) independent trajectories, each respectively of
            length \( n_{i} \), from \( X \) starting from a common
            starting point \( X_0^1 = X_0^2 = \cdots = X_0^m = \sigma_0 \).
        \item
            Let \( v :  \mathcal{X} \to \Reals \) be a value function.
        \item
            Define the random variable \( \mathcal{N} \) to be the
            number of trajectories \( \mathbf{X}^i \) on which \( \sigma_0
            \) is an \( \epsilon \)-outlier.
        \item
            Suppose \( \sigma_0 \) is not an \( (\epsilon, \alpha) \)-outlier.
    \end{enumerate}
    Then
    \[
        \Prob{\mathcal{N} \ge m \sqrt{\frac{\epsilon}{\alpha}} + r} \le
        \EulerE^{r^2/( m \sqrt{\alpha/(\epsilon)} + r)}.
    \]
\end{theorem}

\begin{remark}
    Observe carefully the difference in conclusions of this Theorem~%
    \ref{thm:parallelsignificance:thm6point1} from Theorem~%
    \ref{thm:parallelsignificance:thm3point1} is the absence of the
    factor \( 2 \), so Theorem~%
    \ref{thm:parallelsignificance:thm6point1} is an improvement in the
    probability estimates.  The improvement comes at the cost of a more
    complicated set of parallel Markov chains.
\end{remark}

\begin{proof}
    \begin{enumerate}
        \item
            For a \( \pi \)-stationary trajectory \( X_0, \dots, X_n \)
            and a real number \( \mu \), as a modification of a previous
            notation define \( \rho(0; \epsilon, \mu \given \sigma) \)
            to be the probability that \( v(X_j) \) is in the lowest \(
            \epsilon \) fraction of the values \( v(X_0), \dots, v(X_n) \),
            conditioned on the event \( X_0 = \sigma \), where the
            length \( n \) is chosen from a geometric distribution with
            mean \( \mu \) supported on \( \nu = 0,1,2,\dots \).  The
            modification substitutes the geometric distribution
            parameter \( \mu \) for the fixed trajectory length.  Recall
            the definition of the geometric distribution
            \[
                \Prob{n = \nu} = \frac{1}{\mu+1}\left( 1 - \frac{1}{\mu+1}
                \right)^\nu.
            \]
        \item
            \label{enum:parallelchain:eq15} It suffices to prove that if
            \( \sigma_0 \) is not an \( (\epsilon, \sigma) \)-outlier
            with respect to \( n \) drawn from the geometric
            distribution with mean \( \mu \), then
            \[
                \rho(0; \epsilon, \mu \given \sigma) \le \sqrt{\frac{\epsilon}
                {\alpha}}.
            \] Then the proof can proceed just as in Theorem~%
            \ref{thm:parallelsignificance:thm3point1}.
        \item
            To prove the previous assertion, suppose that \( n_1 \) and \(
            n_2 \) are independent random variables which are
            geometrically distributed with mean \( \mu \), and consider
            a \( \pi \)-stationary trajectory \( X_0, \dots, X_{n_1},
            \dots, X_{n_1 +n_2} \) of random length \( n_1 + n_2 \), and
            condition on the event \( X_{n_1} = \sigma \) for some
            arbitrary \( \sigma \in \mathcal{X} \).  Since \( X \) is
            reversible, we can view this trajectory as two independent
            trajectories \( X_{n_1}, X_{n_1 +1}, \dots, X_{n_1 +n_2} \)
            and \( X_{n_1}, X_{n_1 - 1}, \dots, X_0 \) both beginning
            from \( X_{n_1} = \sigma \), of random lengths \( n_2 \) and
            \( n_1 \) respectively.
        \item
            \label{enum:parallelchain:eq16} Letting \( A \) and \( B \)
            be the events \( v(X_{n_1}) \) is an \( \epsilon \)-outlier
            among the lists \( v(X_0 ), \dots, v(X_{n_1}) \) and \( v(X_
            {n_1}), \dots, v(X_{n_1 +n_2}) \) respectively,
            \begin{multline*}
                \rho(0; \epsilon, \mu \given \sigma)^2 = \Prob{A
                \intersect B} \le \Prob{v(X_{n_1}) \text{ is an \(
                \epsilon \)-outlier among} \right.  \\
                \left.  v(X_0 ), \dots, v(X_{n_1}), v(X_{n_1}), \dots, v
                (X_{n_1 +n_2}) \given X_ {n_1} = \sigma}.
            \end{multline*}
            Remember, in this last expression, \( n_1 \) and \( n_2 \)
            are random variables.
        \item
            The assumption the given \( \sigma_0 \in \mathcal{X} \) is
            not an \( (\epsilon, \alpha) \)-outlier gives that for a
            random \( \sigma \sim \pi \),
            \[
                \Prob{\rho(0; \epsilon, \mu | \sigma)} \ge \Prob{\rho(0;
                \epsilon, \mu | \sigma_{0})} \ge \alpha.
            \]
        \item
            \label{enum:parallelchain:eq18} Write
            \begin{align*}
                \alpha \cdot \rho(0; \epsilon, \mu | \sigma)^{2} \le
                \Esub{\sigma \sim \pi}{\rho(0; \epsilon, | \sigma)^{2}}
                &\le \Prob{v(X_{n_1}) \text{ is an \( \epsilon \)-outlier
                among} \right.  \\
                &\quad\left.  v(X_0 ), \dots, v(X_{n_1}), v(X_{n_1}),
                \dots, v(X_{n_1 +n_2})}
            \end{align*}
            where the last inequality follows from step~%
            \ref{enum:parallelchain:eq16}.
        \item
            Now a simple fact about geometric random variables,
            motivated by considering the right-hand side of step~%
            \ref{enum:parallelchain:eq18}. Conditioning on the value for
            the length \( \ell= n_1 + n_2 \) of the trajectory, \( n_1 \)
            is uniformly distributed in the range \( 0, \dots \ell \).
            This is ensured by the geometric distribution, simply
            because for any \( \ell \) and any \( x \in (0, \dots, \ell)
            \), the probability
            \begin{align*}
                &\Prob{ n_1 = x \text{ and } n_2 = \ell-x} \\
                &\quad = \left( \frac{1}{\mu+1}\left( 1 - \frac{1}{\mu+1}
                \right)^x \right) \cdot \left( \frac{1}{\mu+1}\left( 1 -
                \frac{1}{\mu+1} \right)^{\ell-x} \right) \\
                &\quad = \left( \frac{1}{\mu+1} \right)^2 \left( 1 -
                \frac{1}{\mu+1} \right)^\ell
            \end{align*}
            is independent of \( x \).
        \item
            In particular, conditioning on any particular value for the
            length \( \ell = n_1 + n_2 \), the probability \( v(X_{n_1})
            \) is an \( \epsilon \)-outlier on the trajectory is at most
            \( \epsilon \) since \( X_{n_1} \) is a uniformly randomly
            chosen element of the trajectory \( X_0, \dots, X_{n_1 + n_2}
            \).  This part of the proof is exactly the same as the proof
            of the Besag-Clifford Theorem.
        \item
            In particular, for the right hand side of the inequality in
            step~%
            \ref{enum:parallelchain:eq18}
            \begin{multline*}
                \alpha \cdot \rho(\mu; \epsilon | \sigma)^{2} \le \Prob{v
                (X_{n_1}) \text{ is an \( \epsilon \)-outlier among }
                \right.  \\
                \left.  v(X_0), \dots, v(X_{n_1}), v(X_{n_1}), \dots, v(X_
                {n_1 +n_2})} \\
                \le \max_i \Prob{v(X_{n_1}) \text{ is an \( \epsilon \)-outlier
                among} \right.  \\
                \left.  v(X_0 ), \dots, v(X_{n_1}), v(X_{n_1}), \dots, v
                (X_{n_1 +n_2}) \given n_1 + n_2 = \ell}\le \epsilon.
            \end{multline*}
            This establishes the required inequality in step~%
            \ref{enum:parallelchain:eq15} and completes the proof.
    \end{enumerate}
\end{proof}

\begin{remark}
    When \( m \) is not large, using the exact binomial tail in place of
    the Chernoff bound in~\eqref{eq:parallelsignificance:eleven} may be
    sensible.  This gives the following Corollary.
\end{remark}

\begin{corollary}
    Given the hypotheses of Theorem~%
    \ref{thm:parallelsignificance:thm6point1}, then
    \[
        \Prob{\mathcal{N} \ge K} \le \sum\limits_ {\nu=K}^m \binom{m}{\nu}
        \left( \sqrt{\frac{\epsilon} {\alpha}} \right)^k \left( 1 -
        \sqrt{\frac {\epsilon}{\alpha}} \right)^ {m-k}.
    \]
\end{corollary}

\subsection*{Multiple Testing Considerations}

Theorem~%
\ref{thm:parallelsignificance:thm3point1} uses \( m \) chains, each of
which has a probability of having the starting state \( \sigma_0 \)
being an \( \epsilon \)-outlier where \( \epsilon \) is chosen small.
For any one specific chain, the probability of erroneously rejecting the
null hypothesis is small.  However, the probability of \emph{at least
one} false report is much higher. Performing several tests, then
allowing a significant result on any of them to be evidence for
rejecting the hypothesis requires a multiple testing correction.
Multiple testing%
\index{multiple testing}
also called multiple comparisons, arises when a statistical analysis
involves multiple simultaneous statistical tests, each of which has a
potential to produce a ``discovery'', on the same data set or dependent
data sets. Specifically, there is a chance that the hypothesis which is
actually true is rejected by the test, a Type I error.  The error
probability increases with multiple testing.  A stated confidence level
generally applies only to each test considered individually, but often
it is desirable to have a confidence level for the whole family of
simultaneous tests.  Failure to compensate for multiple comparisons can
have important consequences, see the exercises.

The wrong way to apply Theorem~%
\ref{thm:parallelsignificance:thm3point1} directly with multiple Markov
chains is the following.  Run \( m \) trajectories, record the minimum \(
\epsilon_i \) for which the starting state \( \sigma_0 \) is an \(
\epsilon_i \)-outlier on \( X^i \).  That is, the \( i \)th trajectory
has \( j \) values \( v(X_{\nu_1}^i, \dots v(X_{\nu_j} \) less than \( v
(\sigma_0) \), so \( v(\sigma_0) \) is a \( \epsilon_i = \frac{j+1}{n} \)-outlier.
This creates a list of outlier values \( \epsilon_1 , \epsilon_2, \dots,
\epsilon_ m \).  Then post-hoc, freely choose the parameters \( \alpha \)
and \( \epsilon \) in Theorem~%
\ref{thm:parallelsignificance:thm3point1} to achieve some desired
trade-off between \( \alpha \) and the \( p \)-value.

A solution is to set the parameter \( \epsilon \equiv \epsilon(t) \) as
the \( t \)th-smallest element of the list \( \epsilon_1, \dots,
\epsilon_m \) for some fixed value \( t \).  The case \( t = m \), for
example, corresponds to taking \( \epsilon \) as the maximum value,
leading to the application of Corollary~%
\ref{cor:parallelsignificance:cor32}

The reason this avoids the need for a multiple hypothesis correction is
that hypothesis events are ordered by containment.  Recall the random
variable \( \mathcal{N} \) is the number of trajectories \( \mathbf{X}^i
\) on which \( \sigma_0 \) is an \( \epsilon \)-outlier.  In particular,
applying this test with some value of \( t \), we will always have \(
\mathcal{N} = t \). Thus the \( p \)-value obtained will depend just on
the parameter \( \epsilon_t) \) returned by taking the \( t \)th
smallest \( \epsilon_i \) and on the choice of \( \alpha \).  The \( p \)-value
will not depend on the particular values of the other \( \epsilon_i \)'s
which are not the \( t \)th smallest).  Furthermore, for \( 0 < \alpha
\le 1 \),
\[
    \EulerE^{ -r^{2}/ (2 m\sqrt{2\epsilon/\alpha} + r)} \le \EulerE^{ -r^
    {2}/ (2 m\sqrt{2\epsilon} + r)}
\] and then for \( \epsilon < \epsilon' \),
\[
    \EulerE^{ -r^{2}/ (2 m\sqrt{2\epsilon} + r)} \le \EulerE^{ -r^{2}/ (2
    m\sqrt{2\epsilon'} + r)}.
\] % In particular, regardless of how the values
% \( \alpha \) and \( p \) trade off, the optimum choice of \( \alpha \)
% (for the fixed choice of \( t \)) will depend just on the value
% \( \epsilon(t) \).  In particular, take \( \alpha \) as the optimum
% value
% \( \alpha(\epsilon_t) \) for \( \epsilon_t \).   Applying Theorem~%
% \ref{thm:parallelsignificance:thm3point1} with
% \( \epsilon = \epsilon_t \), evaluates the single-parameter infinite
% family of hypotheses \( H_{\epsilon (t) \alpha(\epsilon(t)) } \), and
Multiple testing correction is not required since the hypotheses are
nested.  That is, \( \epsilon \le \epsilon' \) implies the probability
of erroneously rejecting the null hypothesis with Theorem~%
\ref{thm:parallelsignificance:thm3point1} is less.  Thus, the
probability of returning a \( p \)-value with \( p < p_0 \) for fixed \(
p_0 \) is at most \( p_0 \).

% H_{\epsilon(t),\alpha(\epsilon
%     (t) )} \subseteq H_{\epsilon' (t) ,\alpha(\epsilon(t) )}.
% \] Indeed,
% \[
%     \Prob{ \bigcup_{\epsilon(t) \le \beta } H_{\epsilon(t) ,\alpha(\epsilon
%     (t))} } = \Prob{H_{\beta, \alpha(\beta)}},
% \] which ensures that when applying Theorem~%
% \ref{thm:parallelsignificance:thm3point1} in this scenario, the
% probability of returning a \( p \)-value less than or equal to \( p_0 \)
% for any fixed value \( p_0 \) will indeed be at most \( p_0 \).

\visual{Section Starter Question}{../../../../CommonInformation/Lessons/question_mark.png}
\section*{Section Ending Answer} For \( \epsilon > 0 \), always \(
\epsilon < 2 \epsilon \) so the ``Besag Clifford Test'' is always more
discriminating than the ``Parallel \( 2 \epsilon \) Test'' which in turn
is more discriminating than the ``Serial \( \sqrt{2 \epsilon} \) Test
for \( \epsilon < 1/2 \).  Since \( \epsilon \) will usually be taken to
be one of the traditional statistical significance values of \( p = 0.05
\), \( p = 0.01 \) or \( p = 0.001 \), the parallel tests will always be
better.

\subsection*{Sources} This section is adapted from
\cite{doi:10.1080/2330443X.2020.1806763}.  The definition and examples
of multiple hypothesis correction is adapted from the Wikipedia article
on \link{https://en.wikipedia.org/wiki/Multiple_comparisons_problem}{Multiple
comparisons problem}

\hr

\visual{Algorithms, Scripts, Simulations}{../../../../CommonInformation/Lessons/computer.png}
\section*{Algorithms, Scripts, Simulations}

\subsection*{Algorithm}

\subsection*{Scripts}

%% \input{parallelsignificance_scripts}

\hr

\visual{Problems to Work}{../../../../CommonInformation/Lessons/solveproblems.png}
\section*{Problems to Work for Understanding}
\renewcommand{\theexerciseseries}{}
\renewcommand{\theexercise}{\arabic{exercise}}

\begin{exercise}
    Express the slight changes in notation:
    \begin{itemize}
        \item
            \( \rho(i; \epsilon, n) \) is the probability that in a \( n
            \)-step stationary trajectory \( X_0, X_1, \dots X_k \), \(
            v(X_i) \) is among the smallest \( \epsilon \)-fraction of
            indices in the list \( v(X_0), v(X_1), \dots, v(X_n) \) and
        \item
            \( \rho(0; \epsilon, k \given \sigma_0) \) is the
            probability that on a trajectory \( \sigma_0 = X_0, X_1,
            \dots, X_n \), \( v(\sigma_0) \) is among the smallest \(
            \epsilon \)-fraction of values in the list \( v(X_0), v(X_1),
            \dots, v(X_k) \)
    \end{itemize}
    in terms of the original notations \( \rho(j; \ell, n) \) and \(
    \rho(j; \ell, n \given \sigma) \).
\end{exercise}
\begin{solution}
    To be in the smallest \( \epsilon \)-fraction of indices in the list
    \( v(X_0), v(X_1), \dots, v(X_n) \) means to be among the least \(
    \lfloor \epsilon (n+1) \rfloor \) values.  That is, in terms of the
    original definitions \( \rho(i; \epsilon, n) = \rho(i; \lfloor
    \epsilon (n+1) \rfloor, n) \) and \( \rho(i; \epsilon, n \given
    \sigma) = \rho(i; \lfloor \epsilon (n+1) \rfloor, n \given \sigma) \).

\end{solution}
\begin{exercise}
    Check the Parallel \( 2 \epsilon \) Test by making some simulations
    of the alternate Ehrenfest urn model.  Use the script to find the
    proportion of trials for which the state \( 0 \) is an \( 0.01 \)-outlier
    for the alternative Ehrenfest urn model with \( N = 51 \) balls and
    path length of the chain \( 199, 399, 599, 999 \) steps when
    executing \( 200, 400, 800 \) and \( 1600 \) trials.  Display the
    results in three charts, one for each \( N \).
\end{exercise}
\begin{solution}

\begin{lstlisting}{language = R}
library(markovchain)

N <- 51; p <- 1/2; q <- 1 - p;
stateNames <- as.character( 0:N )
## Be careful here, because states numbered from 0,
## but R indexes from 1
transMatrix <- matrix(0, N+1,N+1)
transMatrix[1,1] <- q
transMatrix[1,2] <- p
transMatrix[N+1, N  ] <- q
transMatrix[N+1, N+1] <- p
for (row in 2:N) {
    transMatrix[row, row-1] <- ((row-1)/N)*q
    transMatrix[row,row] <- ((N-(row-1))/N)*q + ((row-1)/N)*p
    transMatrix[row,row+1] <- ((N-(row-1))/N)*p
}

startState <- "0"

altEhernfest <- new("markovchain", transitionMatrix=transMatrix,
                 states=stateNames, name="AltEhernfest")


pathLength <- c(199, 399, 599, 999)
nTrials <- c(200, 400, 800, 1600)

eps <- 0.01

results <- matrix(0, length(pathLength), length(nTrials))

for (pL in 1:length(pathLength)) {
for  (nT in 1:length(nTrials)) {

outlierThreshold <- floor( eps * ( 2 * pathLength[pL] + 1) ) + 1
# Note the doubling of path length for both Y and Z chains
nEpsOutlier<- 0

for (i in 1:nTrials[nT]) {
            
    pathY <- rmarkovchain(n=pathLength[pL],
                         object = altEhernfest,
                         t0 = startState)
    pathZ <- rmarkovchain(n=pathLength[pL],
                         object = altEhernfest,
                         t0 = startState)
    fullPath <- c(startState, pathY, pathZ)
    labelFullPath <- as.numeric(fullPath)
    labelStartState <- as.numeric(startState)
    if (!identical(sort(labelFullPath)[outlierThreshold], labelStartState)) {
                nEpsOutlier <- nEpsOutlier + 1
    }
}

probEpsOutlier <- nEpsOutlier/nTrials[nT]

results[pL, nT] <- probEpsOutlier
}
}

results
\end{lstlisting}

\begin{verbatim*}
  > results
      [,1]  [,2]    [,3]     [,4]
[1,] 0.705 0.670 0.66625 0.695625
[2,] 0.970 0.975 0.95750 0.958125
[3,] 0.985 1.000 0.99375 0.995625
[4,] 1.000 1.000 1.00000 1.000000
\end{verbatim*}

\end{solution}

\begin{exercise}
    This exercise expands the simulation of the alternate Ehrenfest urn
    model in \texttt{simEpsAlpha.R}. Extend the path length, or number
    of steps, and the number of trials in order to sharpen the rough
    estimates of \( 0 \) being an \( (\epsilon, \alpha) \)-outlier.
\end{exercise}
\begin{solution}
    In \texttt{epsAlpha.R} change \texttt{pathLength} to \( 100 \)
    steps and \textt{nTrials} to \( 200 \).  The parameter \textt{eps}
    is still \( 0.1 \).  The smallest fraction of values will
    be \( \lfloor \epsilon (n+1) \rfloor = \lfloor 0.1 \cdot 101 \rfloor
    = 10 \) values.  That is, the value under consideration must be in
    the least \( 10 \) of the values in the chain.
    The table gives the approximate (rounded to \( 3 \) places)
    probability of being an \( \epsilon \)-outlier
    starting from the state and the probability of being an
    \(epsilon \)-outlier weighted by the stationary distribution.
    \begin{table}
        \centering
        \begin{tabular}{l|cccccccc}
            State    & 0     & 1     & 2     & 3     & 4     & 5     & 6     & 7     \\ 
                     & 0.985 & 0.640 & 0.095 & 0.005 & 0.000 & 0.000 & 0.000 & 0.000 \\ 
            \( \pi \)& 0.008 & 0.055 & 0.164 & 0.273 & 0.273 & 0.164 & 0.055 & 0.008 \\ 
                     & 0.008 & 0.035 & 0.016 & 0.001 & 0.000 & 0.000 & 0.000 & 0.000 \\ 
        \end{tabular}

    The rough fraction of chains starting from a random state from the
    stationary distribution and resulting in the starting state being an
    \( \epsilon \)-outlier is \( 0.008 + 0.035 + 0.016 + 0.001 = 0.06 \).
    The rough fraction of chains starting from a random state from the
    stationary distribution other than \( 0 \) and resulting in the
    starting state being an \( \epsilon \)-outlier is \( 0.035 + 0.016
    + 0.001 = 0.052 \).  This means that among all states \( 0, \dots, 7 \),
    \( 0 \) is more likely than all but about a \( 0.05 \)-fraction of
    states to have an \( v \)-value in the least \( 0.1 \)-fraction of
    values.  Therefore, from this less rough
    example, \( 0 \) is an \( (0.1, 0.05) \)-outlier.
   
\end{solution}

\begin{exercise}
    Using a simple internet search, find examples of multiple testing
    where incorrect conclusions lead to potentially harmful conclusions.
\end{exercise}
\begin{solution}
    The following examples are taken from the Wikipedia article \link{https://en.wikipedia.org/wiki/Multiple_comparisons_problem}
    {Mutliple Comparisons problem.}
    \begin{itemize}
        \item
            Suppose the treatment is a new way of teaching writing to
            students, and the control is the standard way of teaching
            writing.  Students in the two groups can be compared in
            terms of grammar, spelling, organization, content, and so
            on.  These writing attributes might plausibly give dependent
            data sets.  As more attributes are compared, it becomes
            increasingly likely the treatment and control groups will
            differ on at least one attribute due to random sampling
            error alone.
        \item
            Consider the efficacy of a drug in terms of the reduction of
            any one of a number of disease symptoms.  As more symptoms
            are considered, it becomes increasingly likely that the drug
            will appear to be an improvement over existing drugs in
            terms of at least one symptom, again due to random sampling
            error alone.
    \end{itemize}
    In both examples, as the number of comparisons increases, it becomes
    more likely that the groups being compared will appear to differ in
    terms of at least one attribute.  Confidence that a result will
    generalize to independent data should generally be weaker if it is
    observed as part of an analysis that involves multiple comparisons,
    rather than an analysis that involves only a single comparison.

\end{solution}
\hr

\visual{Books}{../../../../CommonInformation/Lessons/books.png}
\section*{Reading Suggestion:}

\bibliography{../../../../CommonInformation/bibliography}

%   \begin{enumerate}
%     \item
%     \item
%     \item
%   \end{enumerate}

\hr

\visual{Links}{../../../../CommonInformation/Lessons/chainlink.png}
\section*{Outside Readings and Links:}
\begin{enumerate}
    \item
    \item
    \item
    \item
\end{enumerate}

\section*{\solutionsname} \loadSolutions

\hr

\mydisclaim \myfooter

Last modified:  \flastmod

\end{document}

%%% Local Variables:
%%% TeX-master: t
%%% TeX-master: t
%%% TeX-master: t
%%% End:
